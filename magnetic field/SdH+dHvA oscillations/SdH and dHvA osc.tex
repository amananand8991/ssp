\documentclass{article}
\usepackage{graphicx} % Required for inserting images
\usepackage{amsmath}

\title{SdH and dHvA oscillations}
\author{Aman Anand}
\date{August 2024}

\begin{document}

\maketitle

\section{Introduction}
Shubnikov-de Haas (SdH) oscillations are oscillations of magnetoresistivity (resistivity under magnetic field) with change in magnetic field (H). de Haas-van Alphen (dhvA) oscillations are oscillations of magnetization (magnetic moment per unit volume) with change in magnetic field (H). These oscillations are periodic in 1/H. There are other oscillatory signatures seen in thermal conductivity, specific heat, optical conductivity, etc.
\subsection{History}
\textbf{From Shoenberg book}\\
Drude's theory explained Wiedemann-Franz law ($\kappa /\sigma \propto T$) but the proportionality constant was off. The paramagnetic susceptibility calculated by taking electron spin into consideration gave large susceptibility proportional to 1/T. This problem rectifies if we use Fermi-Dirac (FD) statistics (Pauli applied FD to susceptibility problem in 1927) and so the susceptibility becomes small and independent of temperature (like the experimental values). Landau (1930) showed diamagnetic susceptibility of free electrons is one third of the Pauli paramagnetic susceptibility. In the same paper Landau showed magnetization should oscillate on varying the field at low temperatures. Onsager (1952)  showed semiclassically that the period of oscillation depended on the extremal area of fermi-surface. It could also be seen as flux through the orbit must be an integer multiple of the flux quanta. 
\begin{equation}
	F (\mathrm{frequency \; of \; oscillations; \; has \; dimesnions \; of \; H}) = (c\hbar/2\pi e)A
\end{equation}


\end{document}
 