\documentclass{article}
\usepackage{graphicx} % Required for inserting images
\usepackage{amsmath}

\title{Crystal and Symmetries}
\author{Aman Anand}
\date{August 2024}

\begin{document}
	
	\maketitle
	
	\section{Introduction}
	Take a lattice with point group symmetry operations \{G\} leaving it invariant. The group elements are $3 \times 3$ unitary matrices. Unlike Bloch translation symmetry which simplifies our problem significantly, these point group symmetries can reduce computations by a factor of 3 or 24. Major significance is in getting the selection rules, as in which transitions are allowed in presence of external potentials (which can cause transition between electron states). 
	
	\section{Mathematics Preliminary}
	What we want to study is how functions (can be wavefunctions) transform under action of the group, i.e.,
	\begin{equation}
		\psi_i (G\mathbf{r}) = \sum_j A(G)_{ij} \psi_j (\mathbf{r}) .
	\end{equation}

	This way we can find the matrices A(G). The set of matrices \{A\} is called a \textbf{representation of the group}. The matrices A and the group elements can all be taken to be unitary $(A^\dagger = A^{-1}$ and $G^\dagger = G^{-1})$. \\
	Two representations $\{A^{(n)}\}$ and $\{A^{(m)}\}$ are \textbf{equivalent} if there exists a single matrix M such that the below holds for all elements of the group.
	\begin{equation}
		M^{-1} A^{(n)} (G) M =  A^{(m)}(G)
	\end{equation} 
	Also, if we can decompose all matrices of our representation in some basis to Block diagonal form, then our representation is called \textbf{reducible}. To derive the main results like the great orthogonality theorem, we consider a matrix M, which is defined by a matrix X (choice of X could be arbitrary)
	\begin{equation}
		M = \sum_G  A^{(m)}(G) X  A^{(n)}(G^\dagger).
	\end{equation}
	Then we see that
	\begin{equation}
		\begin{split}
			 A^{(m)}(G) M &= \sum_{G'} A^{(m)}(G) A^{(m)}(G') X  A^{(n)}(G'^\dagger)\\
			 &= \sum_{G'} A^{(m)}(GG') X  A^{(n)}(G'^\dagger)\\
			 &= \sum_{G'} A^{(m)}(G') X  A^{(n)}(G'^\dagger G) \quad (\mathrm{replace} \; G' \to G^\dagger G')\\
			  &= \sum_{G'} A^{(m)}(G') X  A^{(n)}(G'^\dagger)A^{(n)}( G) = MA^{(n)}.
		\end{split}
	\end{equation}
	So, for $\{A^{(n)}\}$ and $\{A^{(m)}\}$ representations to not be equivalent; the matrix M must be non-invertible. In the same way we can take a conjugate transpose of this to get a relation for matrix $M^\dagger$. We define a matrix, now $P = M + M^\dagger$, this matrix P now also satisfies
	\begin{equation}
		A^{(m)}(G) P = P A^{(n)}(G) 
	\end{equation}
	where P is now a hermitian matrix (as $P^\dagger = P$). Since, P is Hermitian, we can now diagonalize it and find the orthonormal basis. Then we write all the matrices $\{A^{(n)}\},\{A^{(m)}\}$ and P in this basis.\\\\
	\textbf{Q.} How do we know if our representations are irreducible?\\
	If even a single off diagonal term in our representation is 0, then it's reducible. We can see that by taking $n=m$ in the equation above in our diagonal basis, that implies
	\begin{equation}
		A(G)_{ij}P_{jj} = P_{ii} A(G)_{ij},
	\end{equation}
	so, either $P_{ii} = P_{jj}$ or $A(G)_{ij} = 0$. If for any $ij$, let's say $A(G)_{12}$ is 0, then that means no transformation can take $G\psi_1$ to $\psi_2$ and vice versa. Hence, the wavefunctions generated by $G\psi_1$ and $G\psi_2$ are disjoint. So, if our \textbf{representation is irreducible then all the eigenvalues of P must be same}.
	
	
	
	
	
	
	
	
	
	
	
	
	
	
	
	
	
	
	
	
	
\end{document}
