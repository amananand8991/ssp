\documentclass{article}
\usepackage{graphicx} % Required for inserting images
\usepackage{amsmath}
%\usepackage[top=3cm,bottom=3cm,left=3.2cm,right=3.2cm,headsep=10pt,a4paper]{geometry}
\title{Crystal and Symmetries}
\author{Aman Anand}
\date{August 2024}

\begin{document}
	
	\maketitle
	
	\section{Introduction}
	Take a lattice with point group symmetry operations \{G\} leaving it invariant. The group elements are $3 \times 3$ unitary matrices. Unlike Bloch translation symmetry which simplifies our problem significantly, these point group symmetries can reduce computations by a factor of 3 or 24. Major significance is in getting the selection rules, as in which transitions are allowed in presence of external potentials (which can cause transition between electron states). Another advantage of symmetry is knowing the points in \textit{k-space} where the energies are degenerate.
	
	How I plan to write these notes is such that I describe first all the mathematical tools required, then describe the point group symmetries and how there are 32 point groups, 14 Bravais lattice and 230 space groups. Later, I show with examples how this crystal and symmetry information comes handy in knowing our objects of interest.
	
	\section{Mathematics Preliminary}
	What we want to study is how functions (can be wavefunctions) transform under action of the group, i.e.,
	\begin{equation}
		\psi_i (G\mathbf{r}) = \sum_j A(G)_{ij} \psi_j (\mathbf{r}) .
	\end{equation}

	This way we can find the matrices A(G). The set of matrices \{A\} is called a \textbf{representation of the group}. The matrices A and the group elements can all be taken to be unitary $(A^\dagger = A^{-1}$ and $G^\dagger = G^{-1})$. \textbf{How??} \\
	Two representations $\{A^{(n)}\}$ and $\{A^{(m)}\}$ are \textbf{equivalent} if there exists a single matrix N such that the below holds for all elements of the group.
	\begin{equation}
		N^{-1} A^{(n)} (G) N =  A^{(m)}(G)
	\end{equation} 
	It makes sense to take equivalent representations as one otherwise by a change of basis we can get infinitely many representations. Also, if we can decompose all matrices of our representation in some basis to Block diagonal form, then our representation is called \textbf{reducible}. To derive the main results like the great orthogonality theorem, we consider a matrix M, which is defined by a matrix X (choice of X could be arbitrary)
	\begin{equation}\label{Eqn:define-M}
		M = \sum_G  A^{(m)}(G) X  A^{(n)}(G^\dagger).
	\end{equation}
	Then we see that
	\begin{equation}\label{Eqn:relation between representations}
		\begin{split}
			 A^{(m)}(G) M &= \sum_{G'} A^{(m)}(G) A^{(m)}(G') X  A^{(n)}(G'^\dagger)\\
			 &= \sum_{G'} A^{(m)}(GG') X  A^{(n)}(G'^\dagger)\\
			 &= \sum_{G'} A^{(m)}(G') X  A^{(n)}(G'^\dagger G) \quad (\mathrm{replace} \; G' \to G^\dagger G')\\
			  &= \sum_{G'} A^{(m)}(G') X  A^{(n)}(G'^\dagger)A^{(n)}( G) = MA^{(n)}.
		\end{split}
	\end{equation}
	So, for $\{A^{(n)}\}$ and $\{A^{(m)}\}$ representations to not be equivalent; the matrix M must be non-invertible. In the same way we can take a conjugate transpose of this to get a relation for matrix $M^\dagger$. We define a matrix, now $P = M + M^\dagger$, this matrix P now also satisfies
	\begin{equation}
		A^{(m)}(G) P = P A^{(n)}(G) 
	\end{equation}
	where P is now a hermitian matrix (as $P^\dagger = P$). Since, P is Hermitian, we can now diagonalize it and find the orthonormal basis. Then we write all the matrices $\{A^{(n)}\},\{A^{(m)}\}$ and P in this basis.\\\\
	\textbf{Q.} How do we know if our representations are irreducible?\\
	If even a single off diagonal term (for all elements in group) in our representation (in the basis defined previously) is 0, then it's reducible. We can see that by taking $n=m$ in the equation above in our diagonal basis, that implies
	\begin{equation}
		A(G)_{ij}P_{jj} = P_{ii} A(G)_{ij},
	\end{equation}
	so, either $P_{ii} = P_{jj}$ or $A(G)_{ij} = 0$. If for any $ij$, let's say $A(G)_{12}$ is 0, then that means no transformation can take $G\psi_1$ to $\psi_2$ and vice versa. Hence, the wavefunctions generated by $G\psi_1$ and $G\psi_2$ are disjoint. So, if our \textbf{representation is irreducible then all the eigenvalues of P must be same}. If the representation is reducible we get irreducible blocks in this basis. If eigenvalues of a 10-dimensional representation are let's say $P_{11} = P_{22} = P_{33} = 1.2, P_{44} = P_{55} = 1.6$ and other eigenvalues are all distinct. Then we have reduced our representation to one 3-dimensional, one 2-dimensional and five 1-dimensional representation.\\\\
	For different non-equivalent  $\{A^{(n)}\}$ and $\{A^{(m)}\}$ representations which are irreducible, it turns out that the Hermitian matrix P has all eigenvalues which are zero.\\ \\
	\textbf{Schur's Lemma:} For two irreps $\{A^{(n)}\}$ and $\{A^{(m)}\}$ if $S A^{(n)}(G) = A^{(m)} S$ for all G in the group $\{G\}$, then either the two representations are equivalent or $S = 0$. If $m = n$ and again we have a S, for which the previous relation holds true for an irrep; then $S \propto 1$.\\\\
	If we take our matrix X with only a single non-zero element at position $\beta \gamma \;(X_{ij} = \delta_{i\beta}\delta_{j\gamma})$, then using Eqn. (\ref{Eqn:define-M})
	\begin{equation}
		M_{\alpha \delta} = \sum_{G,i,j}A^{(m)}(G)_{\alpha i} \delta_{i\beta}\delta_{j\gamma} A^{(n)}(G^\dagger)_{j \delta} = \sum_{G}A^{(m)}(G)_{\alpha \beta}  A^{(n)}(G^\dagger)_{\gamma \delta}.
	\end{equation}
	For the matrix M, we know it satisfies Eqn. (\ref{Eqn:relation between representations}). hence, by Schur's lemma, we can say that all elements of matrix M should be zero, unless $m=n$, in which case the matrix should be a multiple of identity matrix.
	\begin{equation}
		\sum_{G}A^{(m)}(G)_{\alpha \beta}  A^{(n)}(G^\dagger)_{\gamma \delta} = \delta_{mn}\delta_{\alpha \delta} C_{\beta \gamma}
	\end{equation}
	The precise multiple depends only on what matrix X we start with or in other words only on $\beta \gamma$. To find that constant we sum over $alpha$ and set $\delta = \alpha$ and $n = m$. This simplifies our relation as
	\begin{equation}
		\begin{split}
			\implies &\sum_\alpha \sum_{G}A^{(m)}(G)_{\alpha \beta}  A^{(m)}(G^\dagger)_{\gamma \alpha} = \sum_G A^{(m)}(G^\dagger G)_{\gamma \beta} = \ell C_{\beta \gamma}\\
			&=  \sum_G A^{(m)}(G^\dagger G)_{\gamma \beta} = |G| \delta_{\gamma \beta} \implies C_{\beta \gamma} \frac{|G|}{\ell} \delta_{\gamma \beta}
		\end{split}
	\end{equation}
	where $\ell$ is the dimension of the representation $\{A^{(m)}\}$. This gives us our \textbf{grand orthogonality theorem}:
	
\begin{equation}
	\sum_{G}A^{(m)}(G)_{\alpha \beta}  A^{(n)}(G^\dagger)_{\gamma \delta} = \frac{|G|}{\ell} \delta_{mn}\delta_{\alpha \delta} \delta_{\gamma \beta}
\end{equation}
	Now, we can divide the elements in a group into classes (these classes are all disjoint) and elements $i$ and $j$ are in the same class if $gg_i g^{-1} = g_j$ for any $g \in G$. We define a quantity called character of a representation 
	\begin{equation}
		\chi (G) = \mathrm{Tr}[A(G)].
	\end{equation}
	All the group elements in the same class will have the same character (by trace cyclicity property). So, now we write our orthogonality theorem in terms of characters as
	\begin{equation}
		\sum_{G, \alpha , \gamma}A^{(m)}(G)_{\alpha \alpha}  A^{(n)}(G^\dagger)_{\gamma \gamma} = \frac{|G|}{\ell} \sum_{\alpha , \gamma}\delta_{mn}\delta_{\alpha \gamma} = |G| \delta_{nm}
	\end{equation}
	\begin{equation}\label{Eqn:grand-ortho-for-trace1}
		\implies \sum_k N_k \chi^{(m)}(C_k)\chi^{(n)*}(C_k) = |G| \delta_{nm}
	\end{equation}
	where the sum over $k$ is over distinct classes $C_k$. This equation provides a\textbf{ sure-shot test whether a representation is irreducible or not}, i.e., if the sum of trace squared over all matrices of a representation is $|G|$ then and only then the representation is irreducible otherwise it will be greater than $|G|$.another important result (proof in Murnaghan 1938) is that \textbf{the number of classes equals the number of irreps}. With this result we can write the trace orthogonality relation in another form as
	\begin{equation}\label{Eqn:grand-ortho-for-trace2}
		\sum_k N_n \chi^{(k)}(C_n)\chi^{(k)*}(C_m) = |G| \delta_{nm}.
	\end{equation}
	This can be proved by making the matrices $Q_{mk} = \sqrt{N_k / |G|} \chi^{(m)}(C_k)$ and $Q'_{km} = \sqrt{N_k / |G|} \chi^{(m)}(C_k)$ which now satisfy
	\begin{equation}
		\sum_k Q_{mk} Q'_{km} = \delta_{nm}.
	\end{equation}
	Hence, the matrix $Q'$ is inverse of matrix $Q$, so they should multiply to identity as $\sum_k Q'_{mk} Q_{km} = \delta_{nm}$, from which Eqn (\ref{Eqn:grand-ortho-for-trace2}) follows. The characters of irreps can be written in a table, with the class in the rows and the irrep in the column. This representation is called the \textbf{character table}. We know that identity will be a class with only one element, and in all the representations of the element identity it will be represented as the identity matrix. Hence, from Eq. (\ref{Eqn:grand-ortho-for-trace2}) it follows
	\begin{equation}
		\sum_m d_m ^2 = |G|,
	\end{equation}
	where $\chi^{(m)}(E) = d_m$, is the dimension of the representation and the sum over $m$ is over all irreps of the group. Note: It is often possible to find the character table from just the trace orthogonality relations. See Marder (CMP 2010) for an example of finding the character table of the group $D_{3d}$ (crystallographic point group, naming in Sch\"{o}nflies notation).
	
	
	
	
	
	
	
	
	
\end{document}
