\documentclass{article}
\usepackage{graphicx} % Required for inserting images
\usepackage{amsmath}

\title{Drude model}
\author{Aman Anand}
\date{September 2023}

\begin{document}
	
	\maketitle
	
	\section{Introduction}
	
	Drude model considers solid as nucleus fixed but free electrons moving. The momentum of an electron is described by
	\begin{equation}
		\frac{d\mathbf{p}}{dt} = -\frac{\mathbf{p}}{\tau}+\mathbf{f}(t)
	\end{equation}
	
	\section{Plasmons}
	We find that this model supports collective excitations of the whole system, which we call plasmon. First we study the behaviour of this system under a time varying electric field of the form $\mathbf{E} = \mathbf{E}(\omega)e^{-i\omega t}$
. Note a time varying field also creates a time varying magnetic field, but the induced magnetic field is very small because of the factor $v/c$.	
\begin{equation}
	\frac{d\mathbf{p}}{dt} = -\frac{\mathbf{p}}{\tau}-e\mathbf{E}(\omega)e^{-i\omega t}
\end{equation}
	Take $\mathbf{p} = \mathbf{p}(\omega)e^{-i\omega t}$
	\begin{equation}
		\implies -i\omega \mathbf{p}(\omega) = -\frac{\mathbf{p}(\omega)}{\tau}-e\mathbf{E}(\omega)
	\end{equation}
	\begin{equation}
		\mathbf{p}(\omega) = \frac{e\tau\mathbf{E}(\omega)}{i\omega \tau - 1}
	\end{equation}
Since, $\mathbf{j} = -ne\mathbf{p}/m$
\begin{equation}
	\implies \mathbf{j} (\omega) = \frac{ne^2 \tau}{m}\frac{\mathbf{E}(\omega)}{1-i\omega \tau }
\end{equation}
If the space variation of electric field is much larger than the mean free path, then current density at that point satisfies the DC Ohm's law at that point, i.e.,
\begin{equation}
	\mathbf{j}(\mathbf{r},\omega) = \sigma(\omega) \mathbf{E}(\mathbf{r},\omega) 
\end{equation}
This gives us the frequency dependent conductivity as
\begin{equation}
	\sigma(\omega) = \frac{\sigma_0}{1-i\omega \tau }, \quad \sigma_0 = \frac{ne^2 \tau}{m}.
\end{equation}
To find the frequency-dependent dielectric constant, we use Maxwell's equations
\begin{equation}
	-\nabla^2 \mathbf{E} = \nabla \times (\nabla \times \mathbf{E} ) = \nabla \times \left(-\frac{1}{c}\frac{\partial \mathbf{H}}{\partial  t} \right) = \frac{i\omega}{c} \nabla \times \mathbf{H} = \frac{i\omega}{c} \left(\frac{4\pi}{c}\mathbf{j}+ \frac{1}{c}\frac{\partial \mathbf{E}}{\partial  t} \right)
\end{equation}
\begin{equation}
	\implies -\nabla^2 \mathbf{E} = \frac{i\omega}{c} \left(\frac{4\pi}{c}\sigma(\omega) \mathbf{E} - \frac{i\omega}{c}  \mathbf{E}  \right) = \frac{\omega^2}{c^2} \left(1 + \frac{4\pi i \sigma}{\omega}  \right)\mathbf{E}
\end{equation}
This gives us wave equation of the form 
\begin{equation}
	-\nabla^2 \mathbf{E} =  \frac{\omega^2}{c^2}\epsilon (\omega)\mathbf{E}, \quad \epsilon(\omega) = \left(1 + \frac{4\pi i \sigma}{\omega}  \right).
\end{equation}
In the limit of high-frequency $\omega \tau \gg 1$
\begin{equation}
\epsilon(\omega) = 1 - \frac{\omega_p ^2}{\omega^2}, \quad \omega_p ^2 = \frac{4\pi n e^2}{m}.
\end{equation}






\end{document}
