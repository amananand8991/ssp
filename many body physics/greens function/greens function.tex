\documentclass{article}
\usepackage{graphicx} % Required for inserting images
\usepackage{amsmath}

\title{Green's function}
\author{Aman Anand}
\date{September 2023}

\begin{document}

\maketitle

\section{Introduction}
The Hamiltonian is given as $H = H_0 + V$ and it follows
$H |\psi\rangle = i\hbar \frac{\partial \psi}{\partial t}$
\begin{enumerate}
    \item \textbf{Schrodinger:}  This is the normal Schrodinger equation, and here, wavefunction evolves in time, and operators are not time-dependent.
    \begin{equation}
        \psi_S(t) = e^{-iHt} \psi(0) 
    \end{equation}
    \item \textbf{Heisenberg:} Here, the wavefunction is time-independent, but the operators vary with time as below:
    \begin{equation}
        \psi_H(t) = e^{iHt} \psi_S(t) = \psi_S(0)
    \end{equation}
    \begin{equation}
        A_H(t) = e^{iHt}A_S \;e^{-iHt}
    \end{equation}
    \begin{equation}
        \implies i \frac{dA_H(t)}{dt} = [A_H(t), H]
    \end{equation}
    \item \textbf{Interaction:} Here, both the wavefunction and operators vary with time as follows:
    \begin{equation}
        \psi_I(t) = e^{iH_0t} \psi_S(t) = e^{iH_0t} e^{-iHt} \psi_S(0)
    \end{equation}
    \begin{equation}
        A_I(t) = e^{iH_0t}A_S \;e^{-iH_0t}
    \end{equation}
    \begin{equation}
        \implies  i \frac{dA_I(t)}{dt} = [A_I(t), H_0]
    \end{equation}
\end{enumerate}

Unitary evolution as, $\psi_I(t) = U(t) \psi_I(0)$.\\
\begin{equation}
    \frac{\partial U(t)}{\partial t} = -iV(t) U(t)
\end{equation}
where the $V(t) \equiv V_I (t)$, is in the interaction picture. 
\begin{equation}
\begin{split}
     \implies U(t) &= 1-i\int_0^t dt_1 V(t_1) +(-i)^2 \int_0^t dt_1 \int_0^{t_1} dt_2 V(t_1)  V(t_2)+... \\&= \sum_{n=0}^{\infty} (-i)^n \int_0^{t}dt_1\int_0^{t_1}dt_2 \; ... \int_0^{t_n}dt_n V(t_1)  V(t_2)... V(t_n) \\
     &= 1-i\int_0^t dt_1 V(t_1) +\frac{(-i)^2}{2!} \int_0^t dt_1 \int_0^{t} dt_2 \; T[V(t_1)  V(t_2)]+... \\&=1 + \sum_{n=1}^{\infty} \frac{(-i)^n}{n!} \int_0^{t}dt_1\int_0^{t}dt_2 \; ... \int_0^{t}dt_n \; T[V(t_1)  V(t_2)... V(t_n)] \\ &\equiv T\exp \bigg[-i\int_0^t dt_1\; V(t_1)\bigg]
\end{split}
\end{equation}
Note: In the above equations, V(t) is in the interaction picture operator ($V_I(t)$).\\
S-matrix operator is defined as,  $\psi_I(t) = S(t,t') \psi_I(t')$. Then the S-matrix is given as
\begin{equation}
	S(t,t') = T\exp \bigg[-i\int_{t'}^t dt_1\; V(t_1)\bigg]
\end{equation}

\subsection{Example}
\textbf{Q.} Consider a particle in ground state (at time $t'$) of a harmonic oscillator, now add a time dependent force on it and then see what is the probability for it to be in ground state (at time $t$)? (Coleman MBP 2015)\\
\begin{equation}
	H_0(t) = \hbar \omega \bigg( b^\dagger (t) b(t) + \frac{1}{2} \bigg), \quad V_I(t) = b^\dagger (t) \bar{z}(t) + b(t) z(t)
\end{equation}
where the time dependent operators in Heisenberg notation are $b(t) = b e^{-i\omega t}$ and $b^\dagger(t) = b^\dagger e^{i\omega t}$. Then the quantity of interest that we need to calculate is
\begin{equation}
	  \langle 0 | S(t,t') | 0 \rangle = \langle 0 | T\exp \bigg[-i\int_{t'}^t dt_1\; V(t_1)\bigg] | 0 \rangle
\end{equation}

\begin{equation}
	S_N = e^{-i V(t_N)\Delta t_N}e^{-i V(t_{N-1})\Delta t_{N-1}} \cdots  e^{-i V(t_1)\Delta t_1} = \prod_{j=1}^N e^{-i V(t_{j})\Delta t_{j}}
\end{equation}
Define the quantities 
\begin{equation}
	A_r =  b(t_r) z(t_r), \quad 	A_r ^\dagger =  b^\dagger (t_r) \bar{z}(t_r).
\end{equation}

We bring all annihilation operator to the right and creation ones to the left, with the use of below commutators
\begin{equation}
	[A_i, A_j] = [A^\dagger_i, A^\dagger_j] = 0, \quad [A_i, A^\dagger _j] = z(t_i)\bar{z}(t_j) e^{-i\omega (t_i - t_j)}.
\end{equation}
Now, $e^{-i V(t_{j})\Delta t_{j}} = e^{-iA_j \Delta t_{j}-iA^\dagger_j \Delta t_{j} }$ and we use the BCH identities
\begin{equation}
	e^{\hat{\alpha}+\hat{\beta}} = e^{\hat{\alpha}}e^{\hat{\beta}}e^{-\frac{1}{2}[\hat{\alpha},\hat{\beta}]}, \quad  e^{\hat{\alpha}}e^{\hat{\beta}} = e^{\hat{\beta}}e^{\hat{\alpha}}e^{[\hat{\alpha},\hat{\beta}]}.
\end{equation}
when the commutator of $\hat{\alpha}$ and $\hat{\beta}$ is a c-number.
\begin{equation}
	\implies e^{-iA_j \Delta t_{j}-iA^\dagger_j \Delta t_{j} } =e^{-iA^\dagger_j \Delta t_{j} }  e^{-iA_j \Delta t_{j}}e^{-\frac{1}{2}\Delta t_j ^2 z(t_j)\bar{z}(t_j)}
\end{equation}

\begin{equation}
	\implies  e^{-iA_i \Delta t_{i}} e^{-iA^\dagger_j \Delta t_{j} } = e^{-iA^\dagger_j \Delta t_{j} }e^{-iA_i \Delta t_{i}} e^{-[A_i, A^\dagger _j]\Delta t_i \Delta t_j}
\end{equation}
Hence, we can write our $S_N$ as
\begin{equation}
	S_N = e^{-i\sum_r A^\dagger _r \Delta t_r}e^{-i\sum_r A_r \Delta t_r} \mathrm{exp} \bigg[ -\sum _{i \geq j} [A_i, A^\dagger _j] (1-\frac{1}{2}\delta_{ij}) \Delta t_i \Delta t_j \bigg]
\end{equation}
Taking all time intervals to be equal $(\Delta t_i = \Delta \tau)$ and in the limit of $\Delta \tau \to 0$,
 \begin{equation}
 	 \langle 0 | S(t,t') | 0 \rangle =  \lim_{\Delta \tau \to 0}S_N = \lim_{\Delta \tau \to 0} \langle 0 | e^{-i\sum_r A^\dagger _r \Delta \tau}e^{-i\sum_r A_r \Delta \tau} | 0 \rangle \mathrm{exp} \bigg[ -\sum _{i \geq j} [A_i, A^\dagger _j] (1-\frac{1}{2}\delta_{ij}) \Delta \tau ^2 \bigg]
 \end{equation}
Note that the first part of the product becomes one, as all annihilation operators are on right and creation on left.
\begin{equation}
	\begin{split}
			\implies  \langle 0 | S(t,t') | 0 \rangle &= \lim_{\Delta \tau \to 0} \mathrm{exp} \bigg[ -\sum _{i \geq j} z(t_i)\bar{z}(t_j) e^{-i\omega (t_i - t_j)}(1-\frac{1}{2}\delta_{ij})\Delta \tau ^2 \bigg]\\
			&= \mathrm{exp} \bigg[ -\int_{t'}^t d\tau' d\tau \; \bar{z}(\tau)\theta (\tau'-\tau) e^{-i\omega (\tau' - \tau)}z(\tau') \bigg]\\
			&=  \mathrm{exp} \bigg[ -i\int_{t'}^t d\tau' d\tau \; \bar{z}(\tau) G (\tau - \tau')z(\tau') \bigg]
	\end{split}
\end{equation}
Note, that we ignored the term $\delta_{ij}$ as it didn't contribute to the double integration significantly. Thus, our one-particle Green's function is given as
\begin{equation}
	G (\tau - \tau') = -i \theta (\tau'-\tau) e^{-i\omega (\tau' - \tau)}.
\end{equation}
















\end{document}
 